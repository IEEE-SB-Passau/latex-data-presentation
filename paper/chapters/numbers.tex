\section{Presenting Numbers}

Before we dig into the possibilities of presenting data, we want to give some
general advice about numbers.  When presenting data, almost all is about
numbers; it makes an obvious difference, if a number is presented like
$648312786157.546863$ or \num{648312786157.546863}.  With the former it is hard
to determine the dimension of this number, while it is easy with the latter.

A general advice is to group numbers with four or more digits; whether a small
space~(\mintinline{latex}{\,} in \LaTeX), a comma or a dot is used, depends on
the language, country and personal taste; in Germany, the comma is the decimal
mark, while in English it is the dot.  The German norms DIN~1333 and DIN~5008
state that only a small space should be used for grouping; they are only allowed
money amounts.  In English speaking countries one can use a small space or a
comma for grouping.

\subsection{Units}

When presenting results, the numbers should be consistent in terms of units and
their significance; most of the information in this and the following section is
taken from Beyer, Löwe, and Wendler~\cite{BeyerLoeweWendler2016}.  The first and
most general recommendation is to use SI units; an exception of this rule are
time intervals, where the hours or even days are preferable over seconds.

In Computer Science, we often measure in bytes.  The prefixes \emph{kilo} (k),
\emph{mega} (M), \emph{giga} (G) etc.\ are SI prefixes, that means, they always
represent a factor of \num{1000}.  Because of the binary nature of computers, we
sometimes want to use the factor \num{1024} instead; if we do so, the prefixes
\emph{kibi} (KiB), \emph{mibi} (MiB), \emph{gibi} (GiB), etc.\ need to be used.
Be aware of the different basis, which will lead to an error in precision if
used wrong: with the prefix mega it would be \SI{4.9}{\percent} and
\SI{7.4}{\percent} with giga, respectively—a significant error.

\subsection{Significant Digits}

The presentation of numbers needs to follow another rule.  When measuring for
example the run time of a program, it is impossible to get an exact result; in
contrast the number of tasks in the benchmark set or the number of lines of code
are exact numbers.

For inexact numbers and measures, one needs to define the precision of a
measurement and therefore show all results with a constant number of
\emph{significant digits}.  The German norm DIN~1333 defines the significant
digits as the digits beginning with the first non-zero digit until the digit
that gets rounded.  Notice that leading zeros are not counted as significant
digits, whereas trailing zeros do count.  Consider the numbers
\SI{20}{\kilo\meter} and \SI{20000}{\meter}; the former has two significant
digits, while the latter has five.  To get the same number of significant
digits, the former has to be written as \SI{20.000}{\kilo\meter}.  Writing the
number like this, it would imply that the measurement result was between
\SIlist{19999.5;20000.5}{\meter}.

In order to not confuse the reader, a clear statement of the number of
significant digits that is used in each measurement needs to be made.

\subsection{The Package \toolname{siunitx}}

In order to get the number formatting right, a number of packages evolved; most
known is the \toolname{siunitx} package by Joseph Wright~\cite{Wright2016}.  It
was designed for formatting physical quantities and therefore as a much richer
functionality than we need.

When using the command in Listing~\ref{lst:siunitx-options} to load the package,
all things mentioned above are set correctly.

\begin{listing}[H]
\begin{minted}{latex}
\usepackage[group-digits=integer,%
  group-minimum-digits=4,%
  list-final-separator={, and },%
  add-integer-zero=false,%
  free-standing-units,%
  unit-optional-argument,%
  binary-units,%
  detect-weight=true,%
  detect-inline-weight=math,%
]{siunitx}
\end{minted}
\caption{Suggested options when loading \toolname{siunitx}}
\label{lst:siunitx-options}
\end{listing}

With these settings, all numbers enclosed in all \toolname{siunitx} commands
like \mintinline{latex}{\num{}}, \mintinline{latex}{\numlist{}},
\mintinline{latex}{\SI{}{}} will be formatted the
following way:
\begin{itemize}
  \item only the integer numbers will be grouped, that means, the part left of a
    decimal dot
  \item this is only done if the number has at least four digits.
  \item in a \mintinline{latex}{\numlist}, the last number will be separated
    with \enquote{, and }, e.g. \numlist{5;10;42}
    (\mintinline{latex}{\numlist{5;10;42}}).
  \item if we omit the zero before a decimal dot, it will not be added
    automatically~(e.g. \num{.0815} in contrast to
    \num[add-integer-zero=true]{.0815})
  \item the unit macros like \mintinline{latex}{\second} exist also outside of
    \mintinline{latex}{\si} and \mintinline{latex}{\SI}
  \item these macros have an optional argument, so
    \mintinline{latex}{\SI{10}{\second}} and \mintinline{latex}{\second[10]}
    both produce the output \second[10].
  \item the prefixes \emph{kibi}, \emph{mibi} and so on are available, together
    with \mintinline{latex}{\bit} and \mintinline{latex}{\byte}
  \item \toolname{siunitx} will detect the font weight from the context
\end{itemize}

For a detailed explanation of the options and all other available options of the
\toolname{siunitx} package, we refer the reader to the package
documentation~\cite{Wright2016}.
