\section{Presenting Numbers}

Before we dig into the possibilities of presenting data, we want to give some
general advice about numbers.  When presenting data, almost all is about
numbers; it makes an obvious difference, if a number is presented like
$648312786157.546863$ or \num{648312786157.546863}.  With the former it is hard
to determine the dimension of this number, while it is easy with the latter.

A general advice is to group numbers with four or more digits; whether a small
space~(\verb!\,! in \LaTeX), a comma or a dot is used, depends on the language,
country and personal taste; in Germany, the comma is the decimal mark, while in
English it is the dot.  The German norms DIN~1333 and DIN~5008 state that only a
small space should be used for grouping; they are only allowed money amounts.
In English speaking countries one can use a small space or a comma for grouping.

In order to get the number formatting right, a number of packages evolved; most
known is the \toolname{siunitx} package by Joseph Wright~\cite{Wright2016}.  It
was designed for formatting physical quantities and therefore as a much richer
functionality than we need.

\subsection{Units and Significant Digits}

When presenting results, the numbers should be consistent in terms of units and
their significance; most of the following information is taken
from~\cite{BeyerLoeweWendler2016}.  The first and most general recommendation is
to use SI units; an exception of this rule are time intervals, where the hours
or even days are preferable over seconds.

In Computer Science, we often measure in bytes.  The prefixes \emph{kilo} (k),
\emph{mega} (M), \emph{giga} (G) etc.\ are SI prefixes, that means, they always
represent a factor of \num{1000}.  Because of the binary nature of computers, we
sometimes want to use the factor \num{1024} instead; if we do so, the prefixes
\emph{kibi} (KiB), \emph{mibi} (MiB), \emph{gibi} (GiB), etc.\ need to be used.
Be aware of the different basis, which will lead to an error in precision if
used wrong: with the prefix mega it would be \SI{4.9}{\percent} and
\SI{7.4}{\percent} with giga—a significant error.
