\section{Plots}

In this section, we want to focus on different examples of plots.  We restrict
our presentation to \toolname{pgfplots} because it is the tool we use most;
every example can be done with the other tools, respectively.  The raw data that
is used in the examples can be found in the supplementary GitHub
repository\footnoteref{footnote-github} and on the supplementary web
page\footnoteref{footnote-webpage}.

A good idea before starting with the plot generation is to think about the type
of plot that should be used and the colors in it.  Together with the
\toolname{pgfplots} package comes a library called \emph{colorbrewer}; this
library bundles several color schemes for an easy use.  Before choosing one,
one should think about where the plot will be used—printing often requires
different colors than presentation slides.  Nonetheless, in both cases all plots
should be made of the same color scheme.  Information about the available
schemes can be found in the \toolname{pgfplots}
manual~\cite[Sect.~5.2]{Feuersaenger2016}.

\subsection{Automatic Generation with \hologo{LuaLaTeX}}

With the \TeX{} engine \hologo{LuaTeX}, which combines \hologo{pdfTeX} with
OpenType and Unicode support, \hologo{METAPOST}, and Lua, it is possible to
generate tables or plots from input data while using a simple scripting
language.  Although this is also possible with plain \TeX{}, it might be more
convenient to program in a scripting language rather than \TeX{}.  The example
in Table~\ref{tab:lualatex-autogenerated} is generated from a CSV file that was
exported by the unit test coverage report of IntelliJ IDEA.

\begin{luacode}
tex.print("\\begin{table}[!t]")
tex.print("\\centering")
tex.print("\\caption{Autogenerated table with \\hologo{LuaLaTeX}}")
tex.print("\\label{tab:lualatex-autogenerated}")
tex.print("\\begin{tabular}{@{}l"
  .. "S[table-format=2.2,round-mode=figures,round-precision=4]r"
  .. "r@{}}")
tex.print("\\textbf{Package} & \\textbf{Line} & \\textbf{\\#} & \\\\\\midrule")

local zeile, daten
for zeile in io.lines("../examples/coverage-unittests-model-line-coverage.csv") do
  daten = fromCSV(zeile)
  local linepercent, linebar, pkgname
  pkgname = daten[1]
  linepercent = string.format("%.2f", daten[6] / daten[7] * 100)
  linebar = tonumber(string.format("%.4f", daten[6] / daten[7]))

  tex.print("" .. pkgname .. "&"
    .. linepercent .. "\\,\\% & (" .. daten[6] .. "/" .. daten[7] .. ")&"
    .. "\\begin{tikzpicture}"
    .. "\\draw [black,thin] (0,0) rectangle (1.0,0.15);"
    .. "\\tikz \\fill [gruen] (0,0) rectangle (" .. linebar .. ",0.15);"
    .. "\\tikz \\fill [rot] (" .. linebar .. ",0) rectangle (1.0,0.15);"
    .. "\\end{tikzpicture}"
    .. "\\\\")
end

tex.print("\\end{tabular}")
tex.print("\\end{table}")
\end{luacode}

Obviously, this data can change during the development of an application; one
does not want to adjust the table by hand, every time the coverage data changes.
Therefore, we simply parse the CSV file and create the table using
\hologo{LuaTeX}.  An example code for doing this, can be found in
Listing~\ref{lst:lua-autofilled-table}.

\begin{listing}[H]
  \inputminted[firstline=22,lastline=54]{latex}{../examples/lua-autofilled-table.tex}
  \caption{Code snippet to automatically generate a table like
    Table~\ref{tab:lualatex-autogenerated} with \hologo{LuaLaTeX}}
  \label{lst:lua-autofilled-table}
\end{listing}
