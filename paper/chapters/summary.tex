\section{Summary}

With this work, we have given a short overview over the possibilities of data
presentation in a scientific environment using \hologo{LaTeX}.  We gave hints on
the presentation of numbers and units using the package \toolname{siunitx}.  We
also made the reader aware of the importance of significant digits as well as
the differences between English and German typesetting of numbers.

We continued on tables and gave some general rules of designing tables: use
rules as spare as possible and avoid vertical lines completely!  The
\toolname{booktabs} package provides additional macros for the design of tables.
We also focused on number presentation within tables with the help of the
\toolname{siunitx} package.

Afterwards we introduced some tools to create plots: \toolname{Gnuplot},
\toolname{pgfplots}, \toolname{PSTricks}, and \toolname{R/Sweave}.  Finally, we
concluded with some examples on different plotting types using the
\toolname{pgfplots} package.

Furthermore, we provided a large number of literature and references to the
reader for further studies of the topic.
