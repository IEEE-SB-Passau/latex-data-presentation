\section{Motivation}

When evaluating an experiment, one usually has to deal with a large amount of
data.  After deciding, which data should be discussed in the evaluation part of
a seminar paper or a thesis, the main problem is \emph{how} to present this
data. Obviously, one will choose use tables, plots, or other kinds of figures.
Finding out, which type of presentation is the best, depends on the own point of
view; thus, it will not be part of this work.

We will focus on the different ways of presenting the data.  In order to achieve
this, we will use \LaTeX\@.  We show several packages and how to use them; we
also show some tools, e.g., to easily create plots of the data.  Additionally,
we provide some ways to automatically process the data, in order to minimize the
amount of work that needs to be done manually.  The reason for the latter is to
avoid a complex and error-prone processing work flow.

All source code of the examples, as well as of this paper and the presentation,
can be found on GitHub\footnote{%
  \href{https://github.com/stephanlukasczyk/latex-data-presentation}%
    {https://github.com/stephanlukasczyk/latex-data-presentation}}; on a
supplementary web page\footnote{%
  \href{https://research.lukasczyk.me/latex-data-presentation}%
    {https://research.lukasczyk.me/latex-data-presentation}} we also provide all
examples, both the source code and the compiled results.
