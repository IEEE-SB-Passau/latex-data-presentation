\section{Tables}

When it comes to tables, opinions differ very much.  Everybody has an individual
taste of how a table should look like.  Typesetting tables with \hologo{LaTeX}
is no easy work; on the one hand structuring the source code itself is complex,
on the other hand there is a wide variety of packages on CTAN for getting the
job done.  An overview over packages and possibilities is given by Herbert Voß'
German book \enquote{Tabellen mit \LaTeX}~\cite{Voss2010}.

\subsection{General Rules}

Because of the individual preferences when designing tables, we stick to some
general rules that should be respected in any case here.  The first, and most
important rule is to \emph{never} use vertical lines.  They do not help the
reader and, in complex tables, only add an additional level of complexity.  The
same holds for double rules (and of course for every higher number of
rules)~\cite{Fear2016}.

When using \LaTeX{} it is generally a good idea to load the \toolname{booktabs}
package~\cite{Fear2016} and work through its documentation before starting.

Consider Table~\ref{tab:bad-formated} and Table~\ref{tab:well-formated}; the
former violates the rules, the latter follows them.  We think, it is obvious to
the reader, that Table~\ref{tab:well-formated} is easier to read, and more
clearly structured.

\begin{table}
\label{tab:bad-formated}
\caption{A table with bad formatting~(after~\cite{Voss2010})}
\begin{tabular}{|@{}>{\raggedright}%
  p{3.5cm}@{\kern-30pt}>{\footnotesize}r|*{4}{>{\footnotesize}l|}@{}}\hline\hline
  \textit{Alternative 1} & Investitions- & Jahr &&&\\[-2pt]
                         & zeitpunkt     & 2006 & 2007 & 2008 & 2009
  \\\hline\hline
  Einführungszahlungen & 0 & 0 & 0 & 0 & 0 \\\hline
  Lfd. Personalzahlungen & 0 &  7.187 &  7.187 &  7.187 &  7.187 \\\hline
  Lfd. Zahlungen für \newline
  Wartung/Systempflege   & 0 & 13.572 & 13.572 & 13.572 & 13.572 \\\hline\hline
  Gesamtzahlungen für die Investition
                         & 0 & 20.759 & 20.759 & 20.759 & 20.759 \\\hline\hline
  %
  \multicolumn{6}{c}{\rule{0pt}{3ex}\small(Alle Angaben in \euro)}
\end{tabular}
\end{table}

\begin{table}
\label{tab:well-formated}
\caption{A well-formated table~(taken from~\cite{Voss2010})}
\begin{tabular}{@{}>{\raggedright}%
  p{3.5cm}@{\kern-30pt}*{5}{>{\footnotesize}r}@{}}\toprule
  \textit{Alternative 1} & Investitions- & Jahr\\[-2pt]
                         & zeitpunkt     & 2006 & 2007 & 2008 & 2009 \\
  \cmidrule(lr){3-3}\cmidrule(lr){4-4}\cmidrule(lr){5-5}\cmidrule(l){6-6}
  Einführungszahlungen & 0 & 0 & 0 & 0 & 0 \\
  Lfd. Personalzahlungen & 0 &  7.187 &  7.187 &  7.187 &  7.187 \\
  Lfd. Zahlungen für \newline
  Wartung/Systempflege   & 0 & 13.572 & 13.572 & 13.572 & 13.572 \\
     \cmidrule[0.8pt](r){1-2}\cmidrule(lr){3-3}\cmidrule(lr){4-4}
     \cmidrule(lr){5-5}\cmidrule(l){6-6}
  Gesamtzahlungen für die Investition
                         & 0 & 20.759 & 20.759 & 20.759 & 20.759 \\\bottomrule
  %
  \multicolumn{6}{c}{\rule{0pt}{3ex}\small(Alle Angaben in \euro)}
\end{tabular}
\end{table}

Another important rule is to put the caption of the table above it and not
below—as it is done for figures.  The reason is that a reader will first look at
a table's caption to fathom out the table data, whereas she will first
comprehend a figure as a whole and then look at its
caption~\cite{Schlepzig2010,Schlosser2011}.


\subsection{Tuning}
