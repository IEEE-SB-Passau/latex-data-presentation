\section{Tables}

When it comes to tables, opinions differ very much.  Everybody has an individual
taste of how a table should look like.  Typesetting tables with \hologo{LaTeX}
is no easy work; on the one hand structuring the source code itself is complex,
on the other hand there is a wide variety of packages on CTAN for getting the
job done.  An overview over packages and possibilities is given by Herbert Voß'
German book \enquote{Tabellen mit \LaTeX}~\cite{Voss2010}.

\subsection{General Rules}

Because of the individual preferences when designing tables, we stick to some
general rules that should be respected in any case here.  The first, and most
important rule is to \emph{never} use vertical lines.  They do not help the
reader and, in complex tables, only add an additional level of complexity.  The
same holds for double rules (and of course for every higher number of
rules)~\cite{Fear2016}.

When using \LaTeX{} it is generally a good idea to load the \toolname{booktabs}
package~\cite{Fear2016} and work through its documentation before starting.

Consider Table~\ref{tab:bad-formated} and Table~\ref{tab:well-formated}; the
former violates the rules, the latter follows them.  We think, it is obvious to
the reader, that Table~\ref{tab:well-formated} is easier to read, and more
clearly structured.

\begin{table}[!t]
\caption{A table with bad formatting~(after~\cite{Voss2010})}
\label{tab:bad-formated}
\begin{tabular}{|@{}>{\raggedright}%
  p{3.5cm}@{\kern-30pt}>{\footnotesize}r|*{4}{>{\footnotesize}l|}@{}}\hline\hline
  \textit{Alternative 1} & Investitions- & Jahr &&&\\[-2pt]
                         & zeitpunkt     & 2006 & 2007 & 2008 & 2009
  \\\hline\hline
  Einführungszahlungen & 0 & 0 & 0 & 0 & 0 \\\hline
  Lfd. Personalzahlungen & 0 &  7.187 &  7.187 &  7.187 &  7.187 \\\hline
  Lfd. Zahlungen für \newline
  Wartung/Systempflege   & 0 & 13.572 & 13.572 & 13.572 & 13.572 \\\hline\hline
  Gesamtzahlungen für die Investition
                         & 0 & 20.759 & 20.759 & 20.759 & 20.759 \\\hline\hline
  %
  \multicolumn{6}{c}{\rule{0pt}{3ex}\small(Alle Angaben in \euro)}
\end{tabular}
\end{table}

\begin{table}[!t]
\caption{A well-formated table~(taken from~\cite{Voss2010})}
\label{tab:well-formated}
\begin{tabular}{@{}>{\raggedright}%
  p{3.5cm}@{\kern-30pt}*{5}{>{\footnotesize}r}@{}}\toprule
  \textit{Alternative 1} & Investitions- & Jahr\\[-2pt]
                         & zeitpunkt     & 2006 & 2007 & 2008 & 2009 \\
  \cmidrule(lr){3-3}\cmidrule(lr){4-4}\cmidrule(lr){5-5}\cmidrule(l){6-6}
  Einführungszahlungen & 0 & 0 & 0 & 0 & 0 \\
  Lfd. Personalzahlungen & 0 &  7.187 &  7.187 &  7.187 &  7.187 \\
  Lfd. Zahlungen für \newline
  Wartung/Systempflege   & 0 & 13.572 & 13.572 & 13.572 & 13.572 \\
     \cmidrule[0.8pt](r){1-2}\cmidrule(lr){3-3}\cmidrule(lr){4-4}
     \cmidrule(lr){5-5}\cmidrule(l){6-6}
  Gesamtzahlungen für die Investition
                         & 0 & 20.759 & 20.759 & 20.759 & 20.759 \\\bottomrule
  %
  \multicolumn{6}{c}{\rule{0pt}{3ex}\small(Alle Angaben in \euro)}
\end{tabular}
\end{table}

Another important rule is to put the caption of the table above it and not
below—as it is done for figures.  The reason is that a reader will first look at
a table's caption to fathom out the table data, whereas she will first
comprehend a figure as a whole and then look at its
caption~\cite{Schlepzig2010,Schlosser2011}.


\subsection{Tuning}

Presenting a large amount of numbers in one table might not be as easy as it
seems at first.  Table~\ref{tab:number-formats} gives a comparison of various
formats.  In the first column, all numbers are rounded to the same amount of
decimal columns, which is incorrect as described above.  The remaining columns
have all numbers rounded to four significant digits\footnote{Remark: Trailing
zeros are not counted as significant digits here.}.

It is obvious to the reader that it is difficult to get compare the magnitudes
of the numbers in the second column.  The same holds for the scientific
notation.  In both cases, one can not easily distinguish between values such
as~\numlist{123.5;12.35}; they appear similarly but differ by an order of
magnitude.

The right-most column, however, aligns all numbers at the decimal point.  This
makes it easy for the reader to compare the magnitudes of the different numbers.
Note that we drop leading zeros left of the decimal point in order to increase
this effect.  Generally, dropping leading zeros is not recommended, but in
tables with a lot of numbers of different magnitudes it should be considered, as
it increases the readability and makes the comparison between different numbers
easy.

\begin{table}[!t]
\caption{Example table with different number formats (taken
  from~\cite{BeyerLoeweWendler2016})}
\label{tab:number-formats}
\centering
\begin{tabular}{@{}
  S[table-format=7.4,   round-mode=places, round-precision=4,scientific-notation=fixed]
  S[table-format=6.7,   round-mode=figures,round-precision=4,scientific-notation=fixed,table-alignment=right,table-parse-only=true]
  S[table-format=1.3e-1,round-mode=figures,round-precision=4,retain-zero-exponent]
  S[table-format=6.7,   round-mode=figures,round-precision=4]
  @{}}
\multicolumn{1}{p{0.19\linewidth}}{\centering fixed decimal places} &
\multicolumn{1}{p{0.19\linewidth}}{\centering right-aligned} &
\multicolumn{1}{p{0.19\linewidth}}{\centering scientific notation} &
\multicolumn{1}{p{0.19\linewidth}}{\centering decimal-aligned} \\ \midrule
1.2349876e5  & 1.2349876e5  & 1.2349876e5  & 123498.76 \\
1.2349876e4  & 1.2349876e4  & 1.2349876e4  & 12349.876 \\
1.2349876e3  & 1.2349876e3  & 1.2349876e3  & 1234.9876 \\
1.2349876e2  & 1.2349876e2  & 1.2349876e2  & 123.49876 \\
1.2349876e1  & 1.2349876e1  & 1.2349876e1  & 12.349876 \\
1.2349876e0  & 1.2349876e0  & 1.2349876e0  & 1.2349876 \\
1.2349876e-1 & 1.2349876e-1 & 1.2349876e-1 & .12349876 \\
1.2349876e-2 & 1.2349876e-2 & 1.2349876e-2 & .012349876 \\
1.2349876e-3 & 1.2349876e-3 & 1.2349876e-3 & .0012349876 \\
1.2349876e-4 & 1.2349876e-4 & 1.2349876e-4 & .00012349876 \\
9.876123e-4  & 9.876123e-4  & 9.876123e-4  & .0009876123 \\
9.876123e-3  & 9.876123e-3  & 9.876123e-3  & .009876123 \\
9.876123e-2  & 9.876123e-2  & 9.876123e-2  & .09876123 \\
9.876123e-1  & 9.876123e-1  & 9.876123e-1  & .9876123 \\
9.876123e0   & 9.876123e0   & 9.876123e0   & 9.876123 \\
9.876123e1   & 9.876123e1   & 9.876123e1   & 98.76123 \\
9.876123e2   & 9.876123e2   & 9.876123e2   & 987.6123
\end{tabular}
\end{table}

The \toolname{siunitx} package (or other packages, such as \toolname{dcolumn})
automise the formatting of such tables.  With the former, even the correct
rounding of numbers is done automatically.  Consult the documentation of the
\toolname{siunitx} package for more information on the possibilities of table
formatting.

Obviously, tables can only be used for a small number of results, mostly due to
space issues.  In order to present the full set of results to the reader, they
have to be made available in another form, such as a supplementary web page.
