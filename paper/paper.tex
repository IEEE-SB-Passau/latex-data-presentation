\documentclass[conference,a4paper,english]{IEEEtran}

\usepackage[utf8]{inputenc}
\usepackage[T1]{fontenc}
\usepackage{babel}
\usepackage{tgpagella}
\usepackage[scale=0.9]{tgcursor}
\usepackage{graphicx}
\graphicspath{{./img/}}
\usepackage[cmex10]{amsmath}
\usepackage{amssymb}
\usepackage{booktabs}
\usepackage{multirow}
\usepackage{ragged2e}
\usepackage{csquotes}
\usepackage{flushend} 
\usepackage[backend=biber,%
  hyperref,%
  alldates=iso8601,%
  %backref,%
  %backrefstyle=none,%
  style=numeric,%
  sorting=nyt,%
]{biblatex}
\ExecuteBibliographyOptions{doi=false,url=false}
\newbibmacro{string+doiurlisbn}[1]{%
  \iffieldundef{doi}{%
    \iffieldundef{url}{%
      \iffieldundef{isbn}{%
        \iffieldundef{issn}{%
          #1%
        }{%
          \href{https://books.google.com/books?vid=ISSN\thefield{issn}}{#1}%
        }%
      }{%
        \href{https://books.google.com/books?vid=ISBN\thefield{isbn}}{#1}%
      }%
    }{%
      \href{\thefield{url}}{#1}%
    }%
  }{%
    \href{http://dx.doi.org/\thefield{doi}}{#1}%
  }%
}
\DeclareFieldFormat*{title}{\usebibmacro{string+doiurlisbn}{\mkbibquote{#1}}}

\bibliography{../bibliography.bib}
\usepackage[group-digits=integer,%
  group-minimum-digits=4,%
  list-final-separator={, and },%
  add-integer-zero=false,%
  free-standing-units,%
  unit-optional-argument,%
  binary-units,%
  detect-weight=true,%
  detect-inline-weight=math,%
]{siunitx}
%\SendSettingsToPgf
\usepackage{minted}
\setminted{%
  autogobble,%
  breaklines=true,%
  fontsize=\small,%
  linenos=false,%
  %xleftmargin=2em,%
  %xrightmargin=2em,%
}
\usepackage{paralist}
\usepackage{microtype}
\usepackage[disable]{todonotes}
\usepackage{hyperref}
\hypersetup{%
  pdftitle={Professional Data Presentation with LaTeX},%
  pdfauthor={Stephan Lukasczyk},%
  pdfkeywords={siunitx, booktabs, data, presentation, latex, r, sweave,
    pstricks, pgfplots, lua}
  pdfsubject={},%
}

\begin{document}

\title{Professional Data Presentation with \LaTeX}

\author{\IEEEauthorblockN{Stephan Lukasczyk}
  \IEEEauthorblockA{IEEE Student Branch Passau\\
    94032 Passau, Germany\\
    \href{mailto:tex@lukasczyk.me}{tex@lukasczyk.me}}}

\maketitle

% As a general rule, do not put math, special symbols or citations
% in the abstract
\begin{abstract}
  When working on a seminar paper or a thesis, students often do experimentation
  work. Usually, a large amount of data will be collected by this process. In
  order to present this data in their work, they need appropriate tools and
  follow some rules. We present formal aspects, packages and tools that are
  helpful when presenting the evaluation results.
\end{abstract}
\begin{IEEEkeywords}
  siunitx, booktabs, data, presentation, latex, r, seave, pstricks, pgfplots,
  lua
\end{IEEEkeywords}

\section{Motivation}

When evaluating an experiment, one usually has to deal with a large amount of
data.  After deciding, which data should be discussed in the evaluation part of
a seminar paper or a thesis, the main problem is \emph{how} to present this
data. Obviously, one will choose use tables, plots, or other kinds of figures.
Finding out, which type of presentation is the best, depends on the own point of
view; thus, it will not be part of this work.

We will focus on the different ways of presenting the data.  In order to achieve
this, we will use \LaTeX\@.  We show several packages and how to use them; we
also show some tools, e.g., to easily create plots of the data.  Additionally,
we provide some ways to automatically process the data, in order to minimize the
amount of work that needs to be done manually.  The reason for the latter is to
avoid a complex and error-prone processing work flow.

All source code of the examples, as well as of this paper and the presentation,
can be found on GitHub\footnote{%
  \href{https://github.com/IEEE-SB-Passau/latex-data-presentation}%
    {https://github.com/IEEE-SB-Passau/latex-data-presentation}}; on a
supplementary web page\footnote{%
  \href{https://research.lukasczyk.me/latex-data-presentation}%
    {https://research.lukasczyk.me/latex-data-presentation}} we also provide all
examples, both the source code and the compiled results.

In order to understand the examples, it's not necessary to be a professional
\LaTeX{} user.  We try to provide very simple examples that are
self-explanatory.  Although, we restrict our presentation to \LaTeX{}, the rules
are valid in a general context and can be applied to every other typesetting
tool of your choice.  For a deeper introduction into \LaTeX{}, we suggest the
German book \enquote{Einführung in \LaTeX} by Herbert Voß~\cite{Voss2012} or the
well-known \enquote{\hologo{LaTeX2e}-Kurzbeschreibung} by Marco Daniel et
al~\cite{Daniel2015}.  An English introduction can be found in \enquote{The Not
So Short Introduction to \hologo{LaTeX2e}} by Tobias Oetiker et
al~\cite{Oetiker2015}; a more comprehensive work is \enquote{The \LaTeX{}
Companion} by Frank Mittelbach et al~\cite{Mittelbach2008}.



% trigger a \newpage just before the given reference
% number - used to balance the columns on the last page
% adjust value as needed - may need to be readjusted if
% the document is modified later
%\IEEEtriggeratref{8}
% The "triggered" command can be changed if desired:
%\IEEEtriggercmd{\enlargethispage{-5in}}

% references section

% can use a bibliography generated by BibTeX as a .bbl file
% BibTeX documentation can be easily obtained at:
% http://www.ctan.org/tex-archive/biblio/bibtex/contrib/doc/
% The IEEEtran BibTeX style support page is at:
% http://www.michaelshell.org/tex/ieeetran/bibtex/
%\bibliographystyle{IEEEtran}
% argument is your BibTeX string definitions and bibliography database(s)
%\bibliography{IEEEabrv,../bib/paper}
%
% <OR> manually copy in the resultant .bbl file
% set second argument of \begin to the number of references
% (used to reserve space for the reference number labels box)
\printbibliography

% that's all folks
\end{document}


