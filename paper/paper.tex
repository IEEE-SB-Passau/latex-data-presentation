\documentclass[conference,a4paper,english]{IEEEtran}

\usepackage[utf8]{inputenc}
\usepackage[T1]{fontenc}
\usepackage{babel}
\usepackage{tgpagella}
\usepackage[scale=0.9]{tgcursor}
\usepackage[official]{eurosym}
\usepackage{graphicx}
\graphicspath{{./img/}}
\usepackage[cmex10]{amsmath}
\usepackage{amssymb}
\usepackage{booktabs}
\usepackage{multirow}
\usepackage{ragged2e}
\usepackage{csquotes}
\usepackage{flushend} 
\usepackage[backend=biber,%
  hyperref,%
  alldates=iso8601,%
  %backref,%
  %backrefstyle=none,%
  style=numeric,%
  sorting=nyt,%
]{biblatex}
\ExecuteBibliographyOptions{doi=false,url=false}
\newbibmacro{string+doiurlisbn}[1]{%
  \iffieldundef{doi}{%
    \iffieldundef{url}{%
      \iffieldundef{isbn}{%
        \iffieldundef{issn}{%
          #1%
        }{%
          \href{https://books.google.com/books?vid=ISSN\thefield{issn}}{#1}%
        }%
      }{%
        \href{https://books.google.com/books?vid=ISBN\thefield{isbn}}{#1}%
      }%
    }{%
      \href{\thefield{url}}{#1}%
    }%
  }{%
    \href{http://dx.doi.org/\thefield{doi}}{#1}%
  }%
}
\DeclareFieldFormat*{title}{\usebibmacro{string+doiurlisbn}{\mkbibquote{#1}}}

\bibliography{../bibliography.bib}
\usepackage[group-digits=integer,%
  group-minimum-digits=4,%
  list-final-separator={, and },%
  add-integer-zero=false,%
  free-standing-units,%
  unit-optional-argument,%
  binary-units,%
  detect-weight=true,%
  detect-inline-weight=math,%
]{siunitx}
%\SendSettingsToPgf
\usepackage{minted}
\setminted{%
  autogobble,%
  breaklines=true,%
  fontsize=\small,%
  linenos=false,%
  %xleftmargin=2em,%
  %xrightmargin=2em,%
}
\usepackage{paralist}
\usepackage{microtype}
\usepackage[disable]{todonotes}
\usepackage{hyperref}
\hypersetup{%
  pdftitle={A Beginner's Guide to Scientific Data Presentation using LaTeX},%
  pdfauthor={Stephan Lukasczyk},%
  pdfkeywords={siunitx, booktabs, data, presentation, latex, r, sweave,
    pstricks, pgfplots, lua}
  pdfsubject={},%
}

\begin{document}

\title{A Beginner's Guide to Scientific Data Presentation using \LaTeX}

\author{\IEEEauthorblockN{Stephan Lukasczyk}
  \IEEEauthorblockA{IEEE Student Branch Passau\\
    94032 Passau, Germany\\
    \href{mailto:tex@lukasczyk.me}{tex@lukasczyk.me}}}

\maketitle

% As a general rule, do not put math, special symbols or citations
% in the abstract
\begin{abstract}
  When working on a seminar paper or a thesis, students often do experimentation
  work. Usually, a large amount of data will be collected by this process. In
  order to present this data in their work, they need appropriate tools and
  follow some rules. We present formal aspects, packages and tools that are
  helpful when presenting the evaluation results.
\end{abstract}
\begin{IEEEkeywords}
  siunitx, booktabs, data, presentation, latex, r, seave, pstricks, pgfplots,
  lua
\end{IEEEkeywords}

\section{Motivation}

When evaluating an experiment, one usually has to deal with a large amount of
data.  After deciding, which data should be discussed in the evaluation part of
a seminar paper or a thesis, the main problem is \emph{how} to present this
data. Obviously, one will choose use tables, plots, or other kinds of figures.
Finding out, which type of presentation is the best, depends on the own point of
view; thus, it will not be part of this work.

We will focus on the different ways of presenting the data.  In order to achieve
this, we will use \LaTeX\@.  We show several packages and how to use them; we
also show some tools, e.g., to easily create plots of the data.  Additionally,
we provide some ways to automatically process the data, in order to minimize the
amount of work that needs to be done manually.  The reason for the latter is to
avoid a complex and error-prone processing work flow.

All source code of the examples, as well as of this paper and the presentation,
can be found on GitHub\footnote{%
  \href{https://github.com/stephanlukasczyk/latex-data-presentation}%
    {https://github.com/stephanlukasczyk/latex-data-presentation}}; on a
supplementary web page\footnote{%
  \href{https://research.lukasczyk.me/latex-data-presentation}%
    {https://research.lukasczyk.me/latex-data-presentation}} we also provide all
examples, both the source code and the compiled results.

In order to understand the examples, it's not necessary to be a professional
\LaTeX{} user.  We try to provide very simple examples that are
self-explanatory.  Although, we restrict our presentation to \LaTeX{}, the rules
are valid in a general context and can be applied to every other typesetting
tool of your choice.  For a deeper introduction into \LaTeX{}, we suggest the
German book \enquote{Einführung in \LaTeX} by Herbert Voß~\cite{Voss2012} or the
well-known \enquote{\hologo{LaTeX2e}-Kurzbeschreibung} by Marco Daniel et
al~\cite{Daniel2015}.  An English introduction can be found in \enquote{The Not
So Short Introduction to \hologo{LaTeX2e}} by Tobias Oetiker et
al~\cite{Oetiker2015}; a more comprehensive work is \enquote{The \LaTeX{}
Companion} by Frank Mittelbach et al~\cite{Mittelbach2008}.


\section{Presenting Numbers}

\subsection{Grouping}

\subsection{Significant Digits}


\section{Tables}

When it comes to tables, opinions differ very much.  Everybody has an individual
taste of how a table should look like.  Typesetting tables with \hologo{LaTeX}
is no easy work; on the one hand structuring the source code itself is complex,
on the other hand there is a wide variety of packages on CTAN for getting the
job done.  An overview over packages and possibilities is given by Herbert Voß'
German book \enquote{Tabellen mit \LaTeX}~\cite{Voss2010}.

\subsection{General Rules}

Because of the individual preferences when designing tables, we stick to some
general rules that should be respected in any case here.  The first, and most
important rule is to \emph{never} use vertical lines.  They do not help the
reader and, in complex tables, only add an additional level of complexity.  The
same holds for double rules (and of course for every higher number of
rules)~\cite{Fear2016}.

When using \LaTeX{} it is generally a good idea to load the \toolname{booktabs}
package~\cite{Fear2016} and work through its documentation before starting.

Consider Table~\ref{tab:bad-formated} and Table~\ref{tab:well-formated}; the
former violates the rules, the latter follows them.  We think, it is obvious to
the reader, that Table~\ref{tab:well-formated} is easier to read, and more
clearly structured.

\begin{table}
\label{tab:bad-formated}
\caption{A table with bad formatting~(after~\cite{Voss2010})}
\begin{tabular}{|@{}>{\raggedright}%
  p{3.5cm}@{\kern-30pt}>{\footnotesize}r|*{4}{>{\footnotesize}l|}@{}}\hline\hline
  \textit{Alternative 1} & Investitions- & Jahr &&&\\[-2pt]
                         & zeitpunkt     & 2006 & 2007 & 2008 & 2009
  \\\hline\hline
  Einführungszahlungen & 0 & 0 & 0 & 0 & 0 \\\hline
  Lfd. Personalzahlungen & 0 &  7.187 &  7.187 &  7.187 &  7.187 \\\hline
  Lfd. Zahlungen für \newline
  Wartung/Systempflege   & 0 & 13.572 & 13.572 & 13.572 & 13.572 \\\hline\hline
  Gesamtzahlungen für die Investition
                         & 0 & 20.759 & 20.759 & 20.759 & 20.759 \\\hline\hline
  %
  \multicolumn{6}{c}{\rule{0pt}{3ex}\small(Alle Angaben in \euro)}
\end{tabular}
\end{table}

\begin{table}
\label{tab:well-formated}
\caption{A well-formated table~(taken from~\cite{Voss2010})}
\begin{tabular}{@{}>{\raggedright}%
  p{3.5cm}@{\kern-30pt}*{5}{>{\footnotesize}r}@{}}\toprule
  \textit{Alternative 1} & Investitions- & Jahr\\[-2pt]
                         & zeitpunkt     & 2006 & 2007 & 2008 & 2009 \\
  \cmidrule(lr){3-3}\cmidrule(lr){4-4}\cmidrule(lr){5-5}\cmidrule(l){6-6}
  Einführungszahlungen & 0 & 0 & 0 & 0 & 0 \\
  Lfd. Personalzahlungen & 0 &  7.187 &  7.187 &  7.187 &  7.187 \\
  Lfd. Zahlungen für \newline
  Wartung/Systempflege   & 0 & 13.572 & 13.572 & 13.572 & 13.572 \\
     \cmidrule[0.8pt](r){1-2}\cmidrule(lr){3-3}\cmidrule(lr){4-4}
     \cmidrule(lr){5-5}\cmidrule(l){6-6}
  Gesamtzahlungen für die Investition
                         & 0 & 20.759 & 20.759 & 20.759 & 20.759 \\\bottomrule
  %
  \multicolumn{6}{c}{\rule{0pt}{3ex}\small(Alle Angaben in \euro)}
\end{tabular}
\end{table}

Another important rule is to put the caption of the table above it and not
below—as it is done for figures.  The reason is that a reader will first look at
a table's caption to fathom out the table data, whereas she will first
comprehend a figure as a whole and then look at its
caption~\cite{Schlepzig2010,Schlosser2011}.


\subsection{Tuning}


\section{How to Create Plots?}

We will now show several possibilities for plotting, using different
technologies. For each of the following technologies, we give an example to
provide an idea to the reader of how the work with this technology might look
like. The examples are in alphabetical order of the technology name; afterwards
we will only look closer into \texttt{pgfplots}.

\subsection{GNUplot}

\subsection{pgfplots}

\subsection{PSTricks}

\subsection{R and Sweave}

\toolname{R} is a well-known language for statistical computing and
graphics~\cite{Ihaka1998}.  A huge variety of modules is available at the
Comprehensive R Archive Network (CRAN)\footnote{%
  \href{https://cran.r-project.org}{https://cran.r-project.org}}.  One of these
modules is called \toolname{Sweave}, initially developed by Friedrich
Leisch~\cite{Leisch2002}.  It combines \LaTeX{} and \toolname{R} in the style of
literate programming—a concept developed by Donald E\@. Knuth~\cite{Knuth1992}.
A simple introduction was given by Uwe Ziegenhagen in his German article
\enquote{Datenanalyse mit Sweave, \LaTeX{} und R}~\cite{Ziegenhagen2010}.

In order to use \toolname{Sweave}, one needs to write a document like shown in
Listing~\ref{lst:sweave-example}.  By convention, the filename suffix is not
\texttt{*.tex} but \texttt{*.Snw}; this is important, because the \texttt{*.tex}
file will be create automatically.  A syntax highlighting mode is available for
the editor VIM\@.  After writing the file, one needs to start \toolname{R},
e.g., a \texttt{R} interactive shell.  With the command
\texttt{Sweave("filename.Snw")} the \toolname{R} code will be processed and a
\texttt{*.tex} file will be created, which can then be processed by, e.g.,
\hologo{pdfLaTeX}.

The example in Listing~\ref{lst:sweave-example} loads the exchange rates between
\euro{} and \$ from the homepage of the European Central Bank and plots it. The
output can be seen in Figure~\ref{fig:sweave-example}.

\begin{listing}[H]
  \inputminted{latex}{../examples/sweave-example.Snw}
  \caption{Plot the exchange rate between \euro{} and \$ dynamically using
    Sweave}
  \label{lst:sweave-example}
\end{listing}
\begin{figure}[!t]
  \includegraphics[width=\linewidth]{sweave-example}
  \caption{Screenshot of the Sweave example in Lst.~\ref{lst:sweave-example}}
  \label{fig:sweave-example}
\end{figure}


\section{Plots}

\subsection{Scatter Plots}

\subsection{Quantile Plots}

\subsection{Other Common Types}

\subsection{Automated Generation}



% trigger a \newpage just before the given reference
% number - used to balance the columns on the last page
% adjust value as needed - may need to be readjusted if
% the document is modified later
%\IEEEtriggeratref{8}
% The "triggered" command can be changed if desired:
%\IEEEtriggercmd{\enlargethispage{-5in}}

% references section

% can use a bibliography generated by BibTeX as a .bbl file
% BibTeX documentation can be easily obtained at:
% http://www.ctan.org/tex-archive/biblio/bibtex/contrib/doc/
% The IEEEtran BibTeX style support page is at:
% http://www.michaelshell.org/tex/ieeetran/bibtex/
%\bibliographystyle{IEEEtran}
% argument is your BibTeX string definitions and bibliography database(s)
%\bibliography{IEEEabrv,../bib/paper}
%
% <OR> manually copy in the resultant .bbl file
% set second argument of \begin to the number of references
% (used to reserve space for the reference number labels box)
\printbibliography

% that's all folks
\end{document}


